%% LyX 2.3.0 created this file.  For more info, see http://www.lyx.org/.
%% Do not edit unless you really know what you are doing.
\documentclass[english]{achemso}
\usepackage[T1]{fontenc}
\usepackage[latin9]{inputenc}
\usepackage{babel}
\usepackage{amsmath}
\usepackage{amsthm}
\usepackage{graphicx}
\usepackage{setspace}
\doublespacing
\usepackage[unicode=true,pdfusetitle,
 bookmarks=true,bookmarksnumbered=false,bookmarksopen=false,
 breaklinks=false,pdfborder={0 0 1},backref=false,colorlinks=false]
 {hyperref}
\hypersetup{
 urlcolor=black}

\makeatletter

%%%%%%%%%%%%%%%%%%%%%%%%%%%%%% LyX specific LaTeX commands.

\title{Modeling school closures across 35 countries}

%%%%%%%%%%%%%%%%%%%%%%%%%%%%%% Textclass specific LaTeX commands.
\numberwithin{equation}{section}

%%%%%%%%%%%%%%%%%%%%%%%%%%%%%% User specified LaTeX commands.

\usepackage[nomarkers,figuresonly]{endfloat}

\author{Richard K. Belew}
\email{rbelew@ucsd.edu}
\affiliation{Univ. California -- San Diego\\
La Jolla CA 92093}

\author{Cliff Kerr}
\email{ckerr@idmod.org}
\affiliation{Institute for Disease Modeling, Bellevue, WA}

\author{Jasmina Panovska-Griffiths}
\email{j.panovska-griffiths@ucl.ac.uk}
\affiliation{University College London}

\author{Dina Mistry}
\email{dmistry@idmod.org}
\affiliation{Institute for Disease Modeling, Bellevue, WA}

\SectionNumbersOn

\pagestyle{myheadings}
\markboth{Modeling school closures - Draft 28 Oct 20}{Modeling school closures - Draft 28 Oct 20}

% \DeclareBibliographyCategory{cited}
% \AtEveryCitekey{\addtocategory{cited}{\thefield{entrykey}}}
% \usepackage{filecontents}
% \nocite{*}

\makeatother

\begin{document}
\noindent \begin{center}
\emph{Draft: 28 October 2020}
\par\end{center}
\begin{abstract}
COVID-19 continues to spread around the world and modeling plays an
important role in informing policy. An individual-based model called
Covasim has recently been fit to data regarding confirmed cases and
deaths experience in the United Kingdom during the first half of 2020,
and then used it to evaluate alternative intervention strategies there
\cite{jpg20}. We extend this methodology to consider data from 35
other countries, and use a database of international intervention
specifics called the \href{http://covid19-interventions.com/}{COVID-19 CONTROL STRATEGIES LIST}
to retrospectively model interventions employed in these countries.
Because the age distribution of populations is a key feature of the
COVID-19 pandemic and contacts among young people are often age-stratified
and may play an especially important role, we focus here on school
closure interventions.

Individual countries varied considerably in both the dates on which
they imposed school closings, and in the levels (kindergarten, primary,
secondary, university) specified. Critically, the age-stratified sub-populations
supported by Covasim allow fine-grained specification of just those
individuals affected by school closures at each educational level.
Across the 35 countries the model was calibrated to the country's
epidemic, simulations were first run without the intervention, and
data on confirmed cases and deaths was used to fit key Covasim parameters.
Next, specific intervention strategies employed by each country were
converted into specifications for Covasim, and the same simulation
parameters fit to a second model with the interventions considered.
In the 10 countries where there was a significant difference between
models, those incorporating school closures were considerably better
fits than those without. Since both models' parameters are optimized
and evaluated using the same criterion, improved fit with the intervention
model may be taken as evidence that the modeled interventions were
useful, at least in these 10 countries, in describing observed data.
\end{abstract}

\section{Introduction}

\paragraph*{Comparisons between countries are essential for the control of COVID-19
\cite{10.1001/jama.2020.6585}}

\begin{quotation}

Although international comparisons are often disparaged because of
different data quality and fears of the \textquoteleft ecological
fallacy\textquoteright , if done carefully they can play a major role
in our learning what works best for controlling COVID-19. \cite{10.1093/ije/dyaa108}

... the COVID-19 epidemic shows the need for epidemiology to go back
to its roots---thinking about populations. Studying disease occurrence
by person, place and time (often referred to as \textquoteleft descriptive
epidemiology\textquoteright ) is usually taught in introductory courses,
even if this approach is then paid little attention subsequently.
COVID-19 is a striking example of how we can learn a great deal from
comparing countries, states, regions, time trends and persons, despite
of all the difficulties. \cite{10.1093/ije/dyaa108}

\end{quotation}

With travel restrictions as they were in the spring and summer of
2020, levels of migration across national borders are considerably
smaller than that across state or provincial boundries. In some countries,
school closures were ordered only within particular states. For these
reasons, and given the data currently available, only models at the
level of individual countries and interventions ordered nationally
are considered here. 

\begin{quotation}

Predictive models for large countries, such as the US, are even more
problematic because they aggregate heterogeneous subepidemics in local
areas.... Models should also seek to use the best possible data for
local predictions. \cite{10.1001/jama.2020.6585}

\end{quotation}

The goal of this \emph{post hoc} analysis of historical data is to
understand the limits of our modeling tools as we move forward to
use them for predictive tasks.

\section{Results}

\subsection{Incorporating school closure interventions}

The goal of Covasim optimization, with respect to each individual
country, is to minimize the difference between the model and data
about the number of diagnosed cases of, and deaths caused by, COVID.
The objective measure used for optimization is to minimize sum-squared-difference
(SSD) ``fit value'' between both of these. As data regarding deaths
is believed to generally more accurate than that about positive tests,
SSD over death rates is weighted twice that of diagnoses. In the experiments
reported in this section, model calibration was allowed to vary three
parameters: the number of initial infectsion, beta, and in some experiments,
also testing rate.

The basic experiment is to contrast runs with and with the intervention
of school closures modeled. As a first example, Figure \ref{fig:Sample-countries}
shows the curves for New Zealand, Singapore and Malaysia, three countries
with large improvements in model fit. In the case of New Zealand,
fit with both death and diagnosis data was improved considerably;
in Singapore and Malaysia the improvement was primarily with respect
to death data. The New Zealand model with school closures uses a similar
value for beta but a much smaller initial infected (1161 vs. 11450)
and a much larger testing rate (0.063 vs. .001); Singapore reduced
beta slightly (2.64e-3 vs. 4.19e-3), a much smaller initial infected
(1048 vs. 10825), and less testing (0.247 vs. 0.386). The Malaysian
model with school closures was able to fit much better with quite
similar values for all three parameters.

\begin{figure}
\noindent \begin{centering}
\includegraphics[height=0.8\paperheight]{figs/sampleCountries-sm}
\par\end{centering}
\caption{\label{fig:Sample-countries}Sample countries}
\end{figure}

Figure \ref{fig:All-country-statistics} provides complete statistics
for all 35 countries. The countries have been sorted on the difference
in fit value between the two models, one incorporating school closure
interventions and one without these. Taiwain has proven a difficult
country to model, under many experimental conditions. With this exception,
New Zealand, Singapore and Malaysie are at the top of the list. Several
cells have been highlighted with orange when they are at maximal bounded
values, and with yellow when they are at minimum bounds. These will
be discussed further in Section \ref{subsec:Bounding-search}.

\begin{figure}
\noindent \begin{centering}
\includegraphics[width=0.8\paperwidth]{figs/comp-intrv:non-searchData-summary}
\par\end{centering}
\caption{\label{fig:All-country-statistics}All country statistics}
\end{figure}

Figure \ref{fig:Improved-fit} shows the range of differences in fit
value between these two conditions. The model for Taiwan is very poor
in both conditions. For 14 countries, the difference between the two
modesl was small, < 100. Of the remaining 20 countries where there
was a significant difference between models, in 16 cases those incorporating
school closures were better fits than those without.

\begin{figure}
\noindent \begin{centering}
\includegraphics[width=0.75\paperwidth]{figs/improvedFit-testSearch-annote}
\par\end{centering}
\caption{\label{fig:Improved-fit}Improved fit}
\end{figure}

Figure \ref{fig:Anomolous-countries} shows the pairs of models for
three other countries of potential interest: Romania had an improvement
in fit with the incorporation of school closures that is just better
than 100. The Lithuanian model had very good fit without school closures
and was poorer when this was included. Again, no successful models
of Taiwan were found in either case.

\begin{figure}
\noindent \begin{centering}
\includegraphics[height=0.8\paperheight]{figs/otherCountries-sm}
\par\end{centering}
\caption{\label{fig:Anomolous-countries}Other countries}
\end{figure}


\subsection{Model calibration}

\subsubsection{Stochastic repliction}

The question of how stochastic factors within Covasim vary the results
of simulations is an important first question, because no optimizer
can do well if it is based on noisy results.

A series of n=8 replications of the same simulation were run with
the same parameters. Happily, the range of results across multiple
runs shown in Figure \ref{fig:stochRepl}  shows very tight bounds
across the mean value. (The standard deviation bars around diagnoses
is so small has to be barely visible.)

\begin{figure}
\noindent \begin{centering}
\includegraphics[width=0.8\paperwidth]{figs/Austria-reduce}
\par\end{centering}
\caption{\label{fig:stochRepl}Stochastic replication}
\end{figure}

Figure \ref{fig:stochComp} shows the results across all simulation
measures . Again, very consistent results across runs are shown in
every measure.

\begin{figure}
\noindent \begin{centering}
\includegraphics[width=0.8\paperwidth]{figs/Austria-comp}
\par\end{centering}
\caption{\label{fig:stochComp}Comparing stochastic results}
\end{figure}


\subsubsection{Global optimization}

Many modeling efforts now depend on an optimization component that
identifies best values for key parameters.  This means that there
is no guarantee of finding the actual optimal value \cite{Papadimitriou15881}.

All such optimizations generate several new questions: 
\begin{itemize}
\item How much effort should be put into the search? Ie, how much value
is there in extending computational effort towards improved results?
What sort of reward is given for deeper search?
\item Conversely, what makes a solution ``good enough''?
\item Since we may have multiple criteria for good solutions, how can we
determine the Pareto-optimal combination? For example caring to match
both death data and diagnosis data, how do we determine the Pareto
optimal combination?
\end{itemize}

\subsubsection{Search dynamics}

\paragraph{The value surface}

In the experiments reported here we will generate a sample of 2500
trials of particular parameter pairs of initial infection and beta.
The samples were generated by a CMAES search initiated with the bounds
shown and initialized at the location of the red plus sign. 

On the left is a plot showing all sample points with darker blues
indicating better fit values, a red plus sign showing the initial
search parameter, and a green X showing the labeled best value found.You
need to look closely but you can see several good value ``ridges''
that have been sampled more densely. On the right is a more detailed
contour plot showing an interpolated surface across these samples.
It shows a highly texturued, multi-modal region near the discovered
optimum.

\begin{figure}
\noindent \begin{centering}
\includegraphics[height=3in]{figs/values_testSearch_intrv}\includegraphics[height=3in]{figs/contour_testSearch_intrv}
\par\end{centering}
\caption{\label{fig:valueSurface}Value surface}
\end{figure}

The cells highlighted in the table of Figure \ref{fig:All-country-statistics}
correspond to values at or very near bounds imposed on the search,
orange for values of beta or initial infection near the upper bound,
and yell for values near their lower bound.

\paragraph{Convergence to value optimum}

The samples were generated adaptively, towards better solutions across
``populations'' of size 100, in a series of 25 \textquotedbl generations.\textquotedbl{}
The progression of this search process can be tracked according to
several key statistics as they vary over generations.

A first statistic is the model fit value parameter shown In Figure
\ref{fig:valueConverge}. This shows fit value average, together with
standard deviation bars. The later generations 10-25 have been broken
out in an inset graph, the better to see the smaller values. After
dramatic early improvements, model fit does not improve much after
about 12 generations.

\begin{figure}
\noindent \begin{centering}
\includegraphics[height=3in]{figs/val_testSearch_intrv}
\par\end{centering}
\caption{\label{fig:valueConverge}Value convergence}
\end{figure}

Beta leads initialInfections

Figure \ref{fig:parameterConverge} considers the values of beta and
initial infection independently. Values of beta converge more quickly
and have stabilized by about generation 10. Values of initial infection
continue to increase throughout all generations. Note in Figure \ref{fig:valueConverge},
however, that increasing initial infections does not change model
fit much.

\begin{figure}
\noindent \begin{centering}
\includegraphics[width=0.4\paperwidth]{figs/beta_testSearch_intrv}\includegraphics[width=0.4\paperwidth]{figs/infect_testSearch_intrv}
\par\end{centering}
\caption{\label{fig:parameterConverge}Parameter convergence}
\end{figure}

Deaths dominate diagnoses

Figure \ref{fig:deathDiagn} looks at the component measures on which
model fits is evaluated: match with death data and match with diagnosis
data. As expected, because it is weighted more heavily, matching death
data drives the search. It is interesting to note that improvement
with respect to in the two separate values (diagneses and death) is
well correlated across the generations of the search. 

\begin{figure}
\noindent \begin{centering}
\includegraphics[width=0.4\paperwidth]{figs/cum_deaths_mismatch_testSearch_intrv}\includegraphics[width=0.4\paperwidth]{figs/cum_diagnoses_mismatch_testSearch_intrv}
\par\end{centering}
\caption{\label{fig:deathDiagn}Independent deaths and diagnoses}
\end{figure}


\subsubsection{\label{subsec:Bounding-search}Bounding search}

As mentioned earlier, global optimization procedures generally require
bounds within which they are to search. Figure \ref{fig:valueSurface-large}
can be compared to that originally shown in Figure \ref{fig:valueSurface}.
Here a range of initial infection sizes twice as large (50000 vs 25000)
was allowed in the search. Again, the first plot shows similar ridges
leading from the initial starting value, to an optima with about the
same number of initial infections but a smaller beta. A second, vertical
ridge is indicated leading to the very top boundary imposed by the
a priori constraint.

\begin{figure}
\noindent \begin{centering}
\includegraphics[height=3in]{figs/values_Trials_201008-2}\includegraphics[height=3in]{figs/contour_Trials_201008-2}
\par\end{centering}
\caption{\label{fig:valueSurface-large}Values across larger bounded region}
\end{figure}

Back-of-the-envelope calculations regarding possible values for initial
infections\footnote{r0 = number of contacts {*} days infectious {*} infection probability
per contact per day so if we assume r0 = 2, 20 contacts, 8 days infectious,
we get beta = 0.0125 {[}Cliff, 3 Oct{]}} suggests that the second, upper attractor containing the optimum
in this experiment should be discounted as implausible. \footnote{That's why the primary experiments that are the focus of the results
above!} 

There is a clear bifurcation shown in the second contour plot. The
lower upper bound on initial infection (25000) was selected to force
search to include only the lower attractor. 

\subsection{Using data for testing rate}

Rather than asking the optimization process to select a best testing
rate, data from OWID can be used instead. 

Experiments alternating the two possible sources for testing rates,
data or search, are summarized in Figure \ref{fig:testingSearchVData}.
In all cases, allowing search to establish the parameter value produced
better fitting models. This is to be expected, as a new free parameter
has been allowed. 

\begin{figure}
\noindent \begin{centering}
\includegraphics[height=8in]{figs/comp-data:search-intrv-summary}
\par\end{centering}
\caption{\label{fig:testingSearchVData}Testing rate via search vs. data}
\end{figure}

Figure \ref{fig:testingNZ} shows contrasting results for the case
of New Zealand using search versus data for testing rate. The optimized
value for testing rate shows an enormous number of tests performed,
relative to available data. 

\begin{figure}
\noindent \begin{centering}
\includegraphics[width=0.8\paperwidth]{figs/comp-data:search-intrv-NewZealand}
\par\end{centering}
\caption{\label{fig:testingNZ}Testing search vs. data in New Zealand}
\end{figure}

Note that the improvement allowed with testing rate as free parameter
is despite the error with respect to available data having no consequence
in Pareto-optimum calculation, since it is given a zero weight.

\section{Methods}

\subsection{Data sources}

\begin{figure}
\noindent \begin{centering}
\includegraphics[height=3in]{figs/dataSources}
\par\end{centering}
\caption{\label{fig:Data-sources}Data sources}
\end{figure}

\begin{itemize}
\item OurWorldInData \cite{owidcoronavirus}
\item \href{https://github.com/amel-github/covid19-interventionmeasures}{Covid-19 Control Strategies List},
developed by Am�lie Desvars-Larrive and colleagues {[}\href{https://www.csh.ac.at}{Complexity Science Hub Vienna}{]} 
\end{itemize}
\begin{figure}
\noindent \begin{centering}
\includegraphics[height=3in]{figs/educ_intervene-uniq}
\par\end{centering}
\caption{\label{fig:Intervention-dates-and}Intervention dates and educational
levels}

\end{figure}


\subsection{Covasim}

Covasim \cite{Kerr2020.05.10.20097469}

Optuna \cite{10.1145/3292500.3330701}

Tree-Structured Parzen Estimator \cite{10.1145/3377930.3389817}

CMAEvolution \cite{hansen2016cma}

Did not make use of SynthPop

\subsection{Key details}
\begin{itemize}
\item \texttt{self.weights = sc.mergedicts(\{'cum\_deaths':10, 'cum\_diagnoses':5\},
weights)}
\item \texttt{educLevelSize = \{'h':4, 'w':20, 'c':20, 'sk': 20, 'sp': 20,
'ss': 40 , 'su': 80\}}
\item Only countries with populations > 1e6 were considered.
\item Only countries with educational interventions coming after at least
one week of diagnoses and death data were considered. For example,
while both Ghana and Mauritius had school closures, but data on diagnoses
and deaths was not available for a week before the dates of their
closures.
\item Simulation start date picked when number of infections becomes > 50
\item Each educational level (kindergarten, primary, secondary, university)
was treated as a separate ``layer'' by Covasim and captured into
separate age ranges within the total population. This allows interventions
at different educational levels to be treated independently. 
\item Each school closure at any educational level is modeled as a simple
on/off: school closings start on a specified date, infection rate
was set very low (\texttt{ClosedSchoolBeta = 0.02}) for children in
each age group associated with the educational level, and are assumed
to be in effect until an end-of-closure date.
\item In a small number (6 of 35) of cases, the Covid-19 Control Strategies
List contained a specific \emph{end} for the school closure; for these
countries that date was used to end the intervention. Otherwise, a
default \texttt{SchoolOutDate = 2020-06-01} was used.
\end{itemize}

\subsubsection{Age distribution}
\begin{itemize}
\item \texttt{educLevel\_ages = \{'sk': {[}4,5{]}, 'sp': {[}6,10{]}, 'ss': {[}11,18{]},
'su': {[}19,24{]}\}}
\end{itemize}
The distinguished age cohorts associated with school levels is shown
in Figure 

\begin{figure}
\noindent \begin{centering}
\includegraphics[width=0.8\paperwidth]{figs/Malaysia-people}
\par\end{centering}
\caption{\label{fig:PopAgeDist}Population age distribution}
\end{figure}


\subsection{Model calibration}

An outer-loop of software based on Optuna exists to allow ``calibration''
of Covasim models with respect to key parameters. It supports both
TFE and CMAES optimization techniques.

Population size is a convenient way to broaden the search so as to
reduce sampling error. The CMA library has a ``tell-ask'' interface
that allows easy bundling of parallel executions.

We optimize over two parameters, $\beta$ and the number of initial
infections, and in some cases a third, the assumed testing rate. 

NTrial=100 trials were allocated to find the search for parameter
values causing the model to best fit the data.

\section{Next steps}

\subsection{Model validation}
\begin{enumerate}
\item \{lauerReich20\}: Infectious Disease Forecasting for Public Health
\cite{lauerReich20}
\item \{bergmeir18\}:A note on the validity of cross-validation for evaluating
autoregressive time series prediction \cite{bergmeir18}
\item \{bergmeirBenitez12\}: On the use of cross-validation for time series
predictor evaluation \cite{bergmeirBenitez12}
\end{enumerate}

\subsection{Integration of genomic data: molecular epidemiology}
\begin{itemize}
\item Rockett \cite{Rockett:2020uf}
\begin{enumerate}
\item examine the added value of near real-time genome sequencing of SARS-CoV-2
in a subpopulation of infected patients during the first 10 weeks
of COVID-19 containment in Australia and compare findings from genomic
surveillance with predictions of a computational agent-based model
(ABM)
\item based on AceMOD
\item 21 January and 28 March 2020, 1,617 cases of COVID- 19 were diagnosed
and reported to the NSW Ministry of Health. All patients resided in
metropolitan Sydney. 
\end{enumerate}
\end{itemize}

\section{Conclusions}

\subsection{Honest priors}
\begin{itemize}
\item Bounds; bounded optimization
\item testing rates
\item What accuracy can we claim for model parameter estimation?
\end{itemize}

\subsection{Scientific sharing, publishing and open source models}
\begin{itemize}
\item The expressive power of agent-based models allows evaluation of many
intervention strategies 
\item Open source modeling systems like Covasim allow independent model
components to be investigated and incorporated separately 
\item The publishing of full model implementation, like that included with
the {[}Panovska-Griffiths{]} publication is an excellent example 
\item Fast-paced scientific sharing like that demanded by COVID-19 is catalyzed
by such interactions
\end{itemize}
\pagebreak{}

\bibliographystyle{plain}
\bibliography{covid}

\end{document}

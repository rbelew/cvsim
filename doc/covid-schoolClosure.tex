%% LyX 2.3.0 created this file.  For more info, see http://www.lyx.org/.
%% Do not edit unless you really know what you are doing.
\documentclass[english]{achemso}
\usepackage[T1]{fontenc}
\usepackage[latin9]{inputenc}
\usepackage{babel}
\usepackage{amsmath}
\usepackage{amsthm}
\usepackage{graphicx}
\usepackage{setspace}
\doublespacing
\usepackage[unicode=true,pdfusetitle,
 bookmarks=true,bookmarksnumbered=false,bookmarksopen=false,
 breaklinks=false,pdfborder={0 0 1},backref=false,colorlinks=false]
 {hyperref}
\hypersetup{
 urlcolor=black}
\usepackage{breakurl}

\makeatletter

%%%%%%%%%%%%%%%%%%%%%%%%%%%%%% LyX specific LaTeX commands.

\title{Generalized modeling of interventions across 136 countries}
\newcommand{\noun}[1]{\textsc{#1}}
%% Because html converters don't know tabularnewline
\providecommand{\tabularnewline}{\\}

%%%%%%%%%%%%%%%%%%%%%%%%%%%%%% Textclass specific LaTeX commands.
\numberwithin{equation}{section}

%%%%%%%%%%%%%%%%%%%%%%%%%%%%%% User specified LaTeX commands.

\usepackage[nomarkers,figuresonly]{endfloat}

\author{Richard K. Belew}
\email{rbelew@ucsd.edu}
\affiliation{Univ. California -- San Diego, La Jolla CA}

\author{Cliff Kerr}
\email{ckerr@idmod.org}
\affiliation{Institute for Disease Modeling, Bellevue, WA}

\author{Jasmina Panovska-Griffiths}
\email{j.panovska-griffiths@ucl.ac.uk}
\affiliation{University College London}

\author{Dina Mistry}
\email{dmistry@idmod.org}
\affiliation{Institute for Disease Modeling, Bellevue, WA}

\SectionNumbersOn

\pagestyle{myheadings}
\markboth{Generalized models of 136 countries - Draft 25 Dec 20}{Generalized models of 136 countries - Draft 25 Dec 20}

% \DeclareBibliographyCategory{cited}
% \AtEveryCitekey{\addtocategory{cited}{\thefield{entrykey}}}
% \usepackage{filecontents}
% \nocite{*}

\makeatother

\begin{document}
\noindent \begin{center}
\emph{Draft: 25 December 2020}
\par\end{center}
\begin{abstract}
COVID-19 continues to spread around the world and modeling plays an
important role in informing policy \cite{Press1015}. An individual-based
model called Covasim has recently been fit to data regarding confirmed
cases and deaths experience in the United Kingdom during the first
half of 2020, and then used it to evaluate alternative intervention
strategies there \cite{jpg20}. We extend this methodology to consider
data from 136 countries for which intervention data is available to
retrospectively model interventions employed in these countries. Because
the age distribution of populations is a key feature of the COVID-19
pandemic and contacts among young people are often age-stratified
and may play an especially important role, we focus here on school
closure interventions. For 87 of the countries evaluated there is
also data concerning testing rates in each country that can be compared
to other estimates.

A Covasim model was calibrated for each of 136 countries, with each
country's population, age distribution, disease emergence and school
closing dates specified, then calibrated against data concerning COVID-19
diagnosed cases and related deaths. All countries' simulations were
run with and without school interventions. The hypothesis considered
is: Does incorporating the additional knowledge of school interventions
into a model makes it a better predictor of data concerning diagnoses
and deaths?

Because assumptions regarding background testing rates being performed
are critical for interpretation of statistics regarding the number
of diagnosed patients, three conditions were considered: assume the
same \emph{fixed} testing rated for all countries; \emph{searching}
for an estimated testing rate as part of model calibration; and use
\emph{data} for the 87 countries that have this data available. 

An analysis of the global optimization used to calibrate these models
to parameters is considered in detail. General lessons for using and
interpreting automatic parameter calibration methods are suggested.
The main difficulty, of course, is in establishing what makes one
model a \emph{significantly} better predictor. For now at least we
can \emph{not} reject the null hypothesis: As a general rule, adding
school interventions can not be shown to improve the model.
\end{abstract}

\section{Introduction}

\paragraph*{COVID-19 continues to spread around the world and modeling plays
an important role in informing policy \cite{Press1015}. Here we use
a broad range of countries' experiences to identify robust results
common across them. Comparisons between countries are essential for
the control of COVID-19 \cite{10.1001/jama.2020.6585}}

\begin{quotation}

Although international comparisons are often disparaged because of
different data quality and fears of the \textquoteleft ecological
fallacy\textquoteright , if done carefully they can play a major role
in our learning what works best for controlling COVID-19. \cite{10.1093/ije/dyaa108}

... the COVID-19 epidemic shows the need for epidemiology to go back
to its roots---thinking about populations. Studying disease occurrence
by person, place and time (often referred to as \textquoteleft descriptive
epidemiology\textquoteright ) is usually taught in introductory courses,
even if this approach is then paid little attention subsequently.
COVID-19 is a striking example of how we can learn a great deal from
comparing countries, states, regions, time trends and persons, despite
of all the difficulties. \cite{10.1093/ije/dyaa108}

\end{quotation}

\begin{quotation}

Predictive models for large countries, such as the US, are even more
problematic because they aggregate heterogeneous sub-epidemics in
local areas.... Models should also seek to use the best possible data
for local predictions. \cite{10.1001/jama.2020.6585}

\end{quotation}

With travel restrictions as they were in the spring and summer of
2020, levels of migration across national borders were considerably
smaller than that across state or provincial boundaries. Given the
data currently available, only models at the level of individual countries
and interventions ordered nationally are considered here. 

Because the age distribution of populations is a key feature of the
COVID-19 pandemic and contacts among young people are often age-stratified
and may play an especially important role, we focus here on school
closure interventions. Because global optimization of simulation parameters
is a common aspect of many modeling efforts \cite{10.1371/journal.pcbi.1007893},
we evaluate the optimization process implemented using the CMA-ES
process in some detail. The goal of this \emph{post hoc} analysis
of historical data is to understand the limits of our modeling tools
as we move forward to use them for predictive tasks.

\section{Results}

The first section below reports on ``base model'' experiments that
do \emph{not} include school intervention and which use a \emph{fixed}
testing rate (=3.3e-4, the average across all countries with testing
data; cf. Section \ref{subsec:dataTR}). Experiments contrasting models
searching for the testing rate, and ones using data for testing rate
are in Section \ref{subsec:dataTR}. Experiments contrasting the base
model against simulations including school interventions are considered
in Section \ref{subsec:Incorporating-school-closure}. 

\subsection{Base model}

Covasim defines the value of a model's fit to be a weighted sum of
the error between its predicted number of diagnoses and deaths related
to COVID-19 and data for these values. Figure \ref{fig:base-fit}
shows the range of resulting model fits across all 136 countries,
with lower fitness scores assigned to simulations with a better match
of predictions for the number of diagnoses and deaths. There is not
any absolute criterion as to what makes for a ``good'' fit, but
130 countries have fit values less than 500.

\begin{figure}
\noindent \begin{centering}
\includegraphics[width=0.8\paperwidth]{figs/base-modelFit}
\par\end{centering}
\caption{\label{fig:base-fit}Base model fits}
\end{figure}

Figure \ref{fig:base-examples} shows the result of the simulation
for two countries, COL and LAO. For each country, the top two plots
capture model behavior relative to data for the number of diagnoses
across the time interval considered. The bottom plot simply shows
the number of tests that would have been given over time, assuming
the fixed testing rate and country's population; it isn't useful in
this basic model but will provide contrast in Sections \ref{subsec:dataTR}.
COL and LAO were selected as being a particularly good ``fit'' with
the data (COL), and particularly bad (LAO). Note that this should
be considered the ``best fit found'' with respect to the calibration
optimization over the two free parameters, $\beta$ and \noun{initInfect},
the number of initial infections, and using sum-squared-error as the
measure of mismatch. 

\begin{figure}
\noindent \begin{centering}
\includegraphics[width=0.4\paperwidth]{figs/base-COL-fit}\includegraphics[width=0.4\paperwidth]{figs/base-LAO-fit}
\par\end{centering}
\caption{\label{fig:base-examples}Base model: COL and LAO}
\end{figure}


\subsection{Using available data for testing rate\label{subsec:dataTR}}

For 87 countries there exists data on testing rates via OWID. Experiments
using this source for testing rates can be compared to using a fixed
testing rate, and to rates found via calibration.

\subsubsection{Comparing data against fixed testing rate}

Using data-based testing rates vs. the fixed rate is as likely to
improve as harm the model. As shown in Figure \ref{fig:fixData-trRatioVfit},
while there seems a general trend towards increased testing rates
improving fit, the correlation is not strong ($r^{2}=0.37$). 

\begin{figure}
\noindent \begin{centering}
\includegraphics[width=0.8\paperwidth]{figs/compData_201203_fitVtrRatio}
\par\end{centering}
\caption{\label{fig:fixData-trRatioVfit}Model fit related to data vs. fixed
testing rate}
\end{figure}


\subsubsection{Comparing data against calibrated testing rate}

Figure \ref{fig:searchDataTR} shows the log ratio $log(\frac{testRate_{data}}{testRate_{search}})$
of the testing rate from data vs. search for all countries. In most
cases the testing rate identified via search is lower, and sometimes
much lower, than that provided in data; i.e., most countries are below
the line showing the ratio=1.0. 

\begin{figure}
\noindent \begin{centering}
\includegraphics[width=0.8\paperwidth]{figs/searchData_201221_trRatio}
\par\end{centering}
\caption{\label{fig:searchDataTR}Search vs. data for testing rate - all countries}
\end{figure}

Three countries, BGR,CHE and NGA, sampled from across the range of
results, have been highlighted. Comparisons of model results for these
countries is shown in Figure \ref{fig:searchData-examples}, and details
are provided in Table \ref{tab:Search-vs.-data}. Figure \ref{fig:searchData-BGR-comp}
shows how \emph{daily} testing data compares to the constant (average)
testing \emph{rate} identified via search for BGR; note that BGR has
a data/search testing rate ratio near 1.0. Nevertheless search's slightly
higher testing rate allows an improved value due to a better match
with diagnosis data. In the CHE models, using data for testing rates
improved the fit considerably (although barely noticeable in the plotted
curves). With NGA, on the other hand, using a much larger testing
rate than in the data finds only a slightly better model fit.

\begin{figure}
\noindent \begin{centering}
\includegraphics[width=0.8\paperwidth]{figs/searchData_201221_examples}
\par\end{centering}
\caption{\label{fig:searchData-examples}Search vs. data testing rates for
BGR, CHE and NGA}
\end{figure}

\begin{table}
\noindent \begin{centering}
\begin{tabular}{|c|c|c|c|c|c|c|c|c|c|}
\hline 
\multicolumn{2}{|c|}{} & \multicolumn{4}{c|}{Search} & \multicolumn{4}{c|}{Data}\tabularnewline
\hline 
\hline 
ISO3 & trRatio & infect & beta & testrate & value & infect & beta & testrate & value\tabularnewline
\hline 
\hline 
BGR & 1.07E+0 & 1418 & 5.34E-3 & 2.42E-4 & 231 & 1539 & 5.13E-3 & 2.59E-4 & \textbf{131}\tabularnewline
\hline 
CHE & 3.85E+0 & 42012 & 3.70E-3 & 1.12E-4 & 576 & 34946 & 4.00E-3 & 4.33E-4 & 62\tabularnewline
\hline 
NGA & 2.74E-3 & 32249 & 4.62E-3 & 2.59E-3 & 193 & 9740 & 4.81E-3 & 7.10E-6 & 225\tabularnewline
\hline 
\end{tabular}
\par\end{centering}
\caption{\label{tab:Search-vs.-data}Search vs. data for testing rate - example
countries}
\end{table}

\begin{figure}
\noindent \begin{centering}
\includegraphics[width=0.8\paperwidth]{figs/BGR_testRate-searchData}
\par\end{centering}
\caption{\label{fig:searchData-BGR-comp}Testing rate from search vs. data
for BGR}
\end{figure}

Overall, using data-based testing rates with a model improved their
fit somewhat, in comparison to the value found via calibration search;
average improvement \textasciitilde{} 100\footnote{Note that ``improvement'' by data-based models is reflected in negative
differences in value, since smaller values reflect better models.}. As shown in Figure \ref{fig:searchData-trRatioVfit}, while there
seems a general trend towards increased testing rates improving fit,
the correlation is not strong ($r^{2}=0.34$). 

\begin{figure}
\noindent \begin{centering}
\includegraphics[width=0.8\paperwidth]{figs/searchData_201221_trVfit}
\par\end{centering}
\caption{\label{fig:searchData-trRatioVfit}Model fit related to data/search
testing rate}
\end{figure}

Note that the difference between predicted testing rate and data regarding
was not used by the calibration process to guide the search for model
parameters.

\subsection{Incorporating school closure interventions\label{subsec:Incorporating-school-closure}}

The next set of experiments contrast runs with and without school
closure interventions modeled. Figure \ref{fig:intrvn-goodEg} shows
the curves for two countries, ITA and GBR for which incorporation
of school interventions (base runs on the left, interventions on the
right) improves their fit. The dashed lines in the right-hand figures
reflect the beginning and ending dates of the interventions.

\begin{figure}
\noindent \begin{centering}
\includegraphics[width=0.8\paperwidth]{figs/compIntrvn_goodEg}
\par\end{centering}
\caption{\label{fig:intrvn-goodEg}Interventions improving models}
\end{figure}

Figure \ref{fig:Improved-fit} shows the range of differences in fit
value for all countries, from improved fits (left end) to models that
had worse fit (right end). Countries where both models (with or without
school interventions) had very poor fit scores > 1000 have been highlighted
in red. In these cases, both models are considered broken and contrasting
them tells us little. For the large majority of countries (n=120),
the difference between the two models was small, < 100. Figure \ref{fig:intrvn-badEG}
contrasts the results for two other countries, FRA and CAN, for which
adding interventions to the model harmed their fit to data. Details
of all four runs are included in Table \ref{tab:intrvn-examples}.

\begin{figure}
\noindent \begin{centering}
\includegraphics[width=0.75\paperwidth]{figs/compIntrvn_201216_valDiff}
\par\end{centering}
\caption{\label{fig:Improved-fit}Improved fit}
\end{figure}

\begin{figure}
\noindent \begin{centering}
\includegraphics[width=0.8\paperwidth]{figs/compIntrvn_badEg}
\par\end{centering}
\caption{\label{fig:intrvn-badEG}Interventions harming models}
\end{figure}

\begin{table}
\noindent \begin{centering}
\begin{tabular}{|c|c|c|c|c|c|c|}
\hline 
\multicolumn{1}{|c|}{} & \multicolumn{3}{c|}{None} & \multicolumn{3}{c|}{Intervention}\tabularnewline
\hline 
\hline 
ISO3 & infect & beta & value & infect & beta & value\tabularnewline
\hline 
\hline 
CAN & 49979 & 5.12E-3 & 168 & 49916 & 9.11E-3 & \textbf{409}\tabularnewline
\hline 
FRA & 49692 & 5.84E-3 & 581 & 49739 & 1.04E-2 & 853\tabularnewline
\hline 
GBR & 49869 & 6.02E-3 & 525 & 49980 & 1.30E-2 & 274\tabularnewline
\hline 
ITA & 46062 & 5.66E-3 & 632 & 49249 & 1.72E-2 & 232\tabularnewline
\hline 
\end{tabular}
\par\end{centering}
\caption{\label{tab:intrvn-examples}With and without interventions modeled
- example countries}
\end{table}


\subsubsection{Interpretation}

The changes related to modeling school interventions (vs. not) described
above were based on the simplest alternative regarding testing rates,
viz. using the same, fixed test rate (3.3e-4). The same basic pattern
is shown across the two variations on the 87 countries where data
on testing rates was used instead.

While it is tempting to take positive differences in data like that
from ITA and GBR showing that additional modeling effort seems to
improve the model, but consideration of the differences in model parameters
initInfect and beta makes this seem suspect. The changes in initInfect
are all quite small, relative to the upper bound of the allowed calibration
search space. The role of initial boundaries for calibration are discussed
in Section \ref{subsec:Model-calibration}. In all cases beta has
increased in the intervention models.

That no clear impact of modeling school interventions can be observed
is at least somewhat surprising. A key feature of Covasim is its separation
of individuals into layers with distinctly intra-layer connection
networks. As modeled here, school interventions make significant changes
to all four interaction layers (cf. Table \ref{tab:beta-levels}),
and still don't change much.

\subsection{Model calibration\label{subsec:Model-calibration}}

\subsubsection{Global optimization}

\begin{quotation}

Calibration is a complex and dark art and cannot be covered fully
here; many books have been written about it and it continues to be
an area of active research. A good review article about calibrating
agent-based models like Covasim is available here \cite{10.1371/journal.pcbi.1007893}.
Calibration is usually expressed as an optimization problem: specifically,
find a vector of parameters $\theta$ that minimizes the mismatch
between the data D and the model $M(\theta)$. \href{https://github.com/InstituteforDiseaseModeling/covasim/blob/master/docs/tutorials/t7.ipynb}{Covasim Tutorial\#7}

\end{quotation}

Computational complexity theory warns there can be no guarantee of
finding the actual optimum value \cite{Papadimitriou15881}. In the
context of modeling, the use of calibration to identify some parameter
values generates several new questions: 
\begin{enumerate}
\item Since we may have multiple criteria for good solutions, how can we
determine the Pareto-optimal combination? For example caring to match
both death data and diagnosis data, how do we determine the Pareto
optimal combination?
\item How much effort should be put into the search? Ie, how much value
is there in extending computational effort towards improved results?
What sort of reward is given for deeper search?
\item Conversely, what makes a solution ``good enough''?
\end{enumerate}
For most of the experiments reported here, the answers to these are:
\begin{enumerate}
\item A simple weighted sum of of deaths and diagnosis mismatch is used,
with deaths counting twice as heavily
\item A constant search effort (population size=32, 25 generations) was
allowed the CMAEvolution optimizer
\item There is no early/late termination based on the quality of solutions
found
\end{enumerate}
In this section we describe experiments using many more trial and
search ranges focused on a single country, AUS.

\subsubsection{Search dynamics}

In the experiments reported here we will generate a sample of 2500
trials of particular parameter pairs of initial infection and beta.
The samples were generated by a CMAES search initiated with the bounds
shown and initialized at the location of the red plus sign. 

On the left is a plot of the value surface, showing all sample points
with darker blues indicating better fit values, a red plus sign showing
the initial search parameter, and a green X showing the labeled best
value found. Looking closely you can see several good value ``ridges''
that have been sampled more densely. On the right is a more detailed
contour plot showing an interpolated surface across these samples.
It shows a highly textured, multi-modal region near the discovered
optimum.

\begin{figure}
\noindent \begin{centering}
\includegraphics[height=3in]{figs/values_testSearch_intrv}\includegraphics[height=3in]{figs/contour_testSearch_intrv}
\par\end{centering}
\caption{\label{fig:valueSurface}Value surface}
\end{figure}


\paragraph{Convergence to value optimum}

The samples were generated adaptively, towards better solutions across
CMA-ES populations of size 100, in a series of 25 generations. The
progression of this search process can be tracked according to several
key statistics as they vary over generations.

A first statistic is the model fit value parameter shown In Figure
\ref{fig:valueConverge}. This shows fit value average, together with
standard deviation bars. The later generations 10-25 have been broken
out in an inset graph, the better to see the smaller values. After
dramatic early improvements, model fit does not improve much after
about 12 generations.

\begin{figure}
\noindent \begin{centering}
\includegraphics[height=3in]{figs/val_testSearch_intrv}
\par\end{centering}
\caption{\label{fig:valueConverge}Value convergence}
\end{figure}

Beta leads initialInfections

Figure \ref{fig:parameterConverge} considers the values of beta and
initial infection independently. Values of beta converge more quickly
and have stabilized by about generation 10. Values of initial infection
continue to increase throughout all generations. Note in Figure \ref{fig:valueConverge},
however, that increasing initial infections does not change model
fit much.

\begin{figure}
\noindent \begin{centering}
\includegraphics[width=0.4\paperwidth]{figs/beta_testSearch_intrv}\includegraphics[width=0.4\paperwidth]{figs/infect_testSearch_intrv}
\par\end{centering}
\caption{\label{fig:parameterConverge}Parameter convergence}
\end{figure}

Deaths dominate diagnoses

Figure \ref{fig:deathDiagn} looks at the component measures on which
model fits is evaluated: match with death data and match with diagnosis
data. As expected, because it is weighted more heavily, matching death
data drives the search. It is interesting to note that improvement
with respect to in the two separate values (diagnoses and death) is
well correlated across the generations of the search. 

\begin{figure}
\noindent \begin{centering}
\includegraphics[width=0.4\paperwidth]{figs/cum_deaths_mismatch_testSearch_intrv}\includegraphics[width=0.4\paperwidth]{figs/cum_diagnoses_mismatch_testSearch_intrv}
\par\end{centering}
\caption{\label{fig:deathDiagn}Independent deaths and diagnoses}
\end{figure}


\subsubsection{\label{subsec:Bounding-search}Bounding search}

As mentioned earlier, global optimization procedures generally require
bounds within which they are to search. Figure \ref{fig:valueSurface-large}
can be compared to that originally shown in Figure \ref{fig:valueSurface}.
Here a range of initial infection sizes twice as large (50000 vs 25000)
was allowed in the search. Again, the first plot shows similar ridges
leading from the initial starting value, to an optima with about the
same number of initial infections but a smaller beta. A second, vertical
ridge is indicated leading to the very top boundary imposed by the
a priori constraint.

\begin{figure}
\noindent \begin{centering}
\includegraphics[height=3in]{figs/values_Trials_201008-2}\includegraphics[height=3in]{figs/contour_Trials_201008-2}
\par\end{centering}
\caption{\label{fig:valueSurface-large}Values across larger bounded region}
\end{figure}

Back-of-the-envelope calculations regarding possible values for initial
infections\footnote{r0 = number of contacts {*} days infectious {*} infection probability
per contact per day so if we assume r0 = 2, 20 contacts, 8 days infectious,
we get beta = 0.0125 {[}Cliff, 3 Oct{]}} suggests that the second, upper attractor containing the optimum
in this experiment should be discounted as implausible. 

There is a clear bifurcation shown in the second contour plot. The
lower upper bound on initial infection (25000) was selected to force
search to include only the lower attractor. 

\section{Methods}

\subsection{Data sources}

ECDC used as primary organizing data source. ISO3 codes were used
to identify and merge country data.\footnote{Two countries missing from ECDC lists, TWN and HNG. 2019 population
populations for these two was obtained from 'pop19': 23773876 \}\href{https://www.worldometers.info/world-population/}{WorldOMeters} }
\begin{itemize}
\item \href{https://www.ecdc.europa.eu/sites/default/files/documents/COVID-19-geographic-disbtribution-worldwide-2020-08-04.xlsx}{ECDC}
\item \href{ttps://github.com/owid/covid-19-data/blob/master/public/data/testing/covid-testing-all-observations.csv}{OWID testing}\cite{owidcoronavirus}
\item \href{http://en.unesco.org/sites/default/files/covid_impact_education.csv}{UNESCO interventions}
\item \href{http://github.com/OxCGRT/covid-policy-tracker/blob/master/data/OxCGRT_latest_responses.csv}{OxCGRT interventions}
\item Age distribution data from the \href{https://github.com/neherlab/covid19_scenarios/blob/master/src/assets/data/country_age_distribution.json}{Neher Lab}
as distributed as part of Covasim\footnote{cf. \texttt{covasim.data.country\_age\_data.py}}
\end{itemize}
\begin{figure}
\noindent \begin{centering}
\includegraphics[height=3in]{figs/dataSources-v2}
\par\end{centering}
\caption{\label{fig:Data-sources}Data sources}
\end{figure}

Only countries with populations greater than one million were considered,
and the start date for each countries' simulation was picked to begin
when data showed the number of infections went above 50.

\subsubsection{School interventions}

Preliminary experiments used a database of interventions developed
by a database of international intervention specifics called \href{https://github.com/amel-github/covid19-interventionmeasures}{Covid-19 Control Strategies List},
developed by Am�lie Desvars-Larrive and colleagues {[}\href{https://www.csh.ac.at}{Complexity Science Hub Vienna}{]}.
This data includes a fine-grained analysis of which school levels
(kindergarten, primary, secondary, university) were ordered closed,
and therefore supported fine-grained variations in the age distributions
of the sort used effectively by Covasim's population generator. 37
countries were included in this data.

The experiments reported here used instead a \emph{consensus} database
of school closures coming from two distinct data sources, OxCGRT and
UNESCO. While broadly consistent, their respective accounts as to
just when individual countries were closed nationally was not identical.
Figure \ref{fig:overlapIntervnData} shows two countries' examples:
MNG, a country with nearly overlapping accounts; and UZB with more
discordant ones. The graphs each show two plots of when school closures
were in effect (a value of 1, vs. 0 when not in effect) as described
by a source.

\begin{figure}
\noindent \begin{centering}
\includegraphics[width=0.4\paperwidth]{figs/MNG_ox-unesco_educIntrvn}
\includegraphics[width=0.4\paperwidth]{figs/UZB_ox-unesco_educIntrvn}
\par\end{centering}
\noindent \centering{}\caption{\label{fig:overlapIntervnData}Overlapping intervention data}
\end{figure}

The specification of school closures used in these experiments was
conservative: Only intervention periods that were identified by \emph{both}
sources, for periods > 14 days were considered. 136 countries had
school interventions that satisfied these criteria.

\subsubsection{Testing rate data}

Testing data from OurWorldInData \cite{owidcoronavirus} was used.
For the comparisons in Section \ref{subsec:dataTR}, this daily testing
data was converted to an average testing rate using the country's
population and averaging across the time period of the simulation.

Table \ref{fig:countryData} summarizes the 136 countries included
in this study: their name, ISO-3 code, and whether or not they had
testing rate data available about them. 

\begin{figure}
\noindent \begin{centering}
\includegraphics[width=0.8\paperwidth]{figs/dataOverlapTbl}
\par\end{centering}
\caption{\label{fig:countryData}Countries' data}
\end{figure}


\subsection{Covasim}

Covasim \cite{Kerr2020.05.10.20097469} is an open-source agent-based
(a.k.a. individual-based) model that uses demographic information
on age structure and population size to build realistic transmission
networks within distinct social layers. It allows age-specific disease
outcomes, intrahost viral dynamics, and the ability to incorporate
many sorts of interventions affecting these model elements. 

The basic model requires specification of two key parameters, the
initial number of infected individuals in the population, and $\beta$,
the transmission rate from susceptible to infected individuals. The
global $\beta$ parameter is proposed by the calibration optimizer,
then enhanced/attenuated across individual network layers according
to parameters based loosely on time-use surveys that track how many
hours a week people spend in various settings (work, school, transport,
etc), with households getting an additional multiplier for closeness
of contact. 

As data regarding deaths is believed to generally more accurate than
that about positive tests, Covasim uses a weighted sum of error with
death rates weighted twice as heavily as diagnoses.

School closures are modeled as a simple on/off: school closings start/stop
on a specified dates, and layer-specific values for $\beta$ are again
enhanced or attenuated. Following \cite{jpg20}, infection rate was
set very low within the schools level, increased within the home level,
and lowered significantly in work and community levels. The net values
for $\beta$ incorporating both basic Covasim values and the result
of school interventions are shown in Table \ref{tab:beta-levels}.

\begin{table}
\noindent \begin{centering}
\begin{tabular}{|c|c|c|c|}
\hline 
$\beta$  & Covasim & Intervention & Net with intervention\tabularnewline
\hline 
\hline 
Home & 3.00 & 1.29 & 3.87\tabularnewline
\hline 
School & 0.60 & 0.02 & 0.01\tabularnewline
\hline 
Work & 0.60 & 0.20 & 0.12\tabularnewline
\hline 
Community & 0.30 & 0.20 & 0.06\tabularnewline
\hline 
\end{tabular}
\par\end{centering}
\caption{\label{tab:beta-levels}Per-level $\beta$ values}
\end{table}


\subsection{CMA-ES}

Covariance matrix adaptation evolution strategy (CMA-ES) \cite{hansen2016cma}
is a library for stochastic optimization over continuous domains of
non-linear, non-convex functions. The ``ES'' refers to ``Evolutionary
strategy,'' a type of evolutionary computation using the metaphors
of populations and generations, and distinguished by adaptive rates
of mutation across generations. The CMA library has a ``tell-ask''
interface that allows easy bundling of parallel executions of a population
alternative values in each generation. The initial range of mutation
$\sigma_{0}$ was set to 0.1, and the initial search range for all
parameters was set to one quarter of the range between their upper
and lower bounds.

CMA-ES was used as the ``calibration'' outer-loop around the basic
Covasim simulation to search for key parameter values that allow the
model to best ``fit'' against data. Preliminary experiments across
a large number of trials on a single country (cf. Section \ref{subsec:Model-calibration})
identified 25 generations of 32 individuals each as being reasonable;
i.e., 800 trials total were allocated to find the search for parameter
values causing the model to best fit the data. Optimization is always
over two parameters, $\beta$ and \noun{initInfect}, the number of
initial infections. Based on the experiments reported in Section \ref{subsec:Model-calibration},
initial search bounds for the optimization were set:
\begin{itemize}
\item initial infected: best = 21000, lower bound =16000, upper bound=26000
\item $\beta$: best = .005, lower bound = 0.001, upper bound=0.01
\end{itemize}
In the experiments described in Section \ref{fig:searchDataTR}, a
third parameter corresponding to testing rate is also included. Bounds
on its search were established using the data for the 87 countries
available:
\begin{itemize}
\item testingRate: best=3.3e-4, lower bound = 2e-6, upper bound=3e-3
\end{itemize}

\section{Next steps}
\begin{itemize}
\item Use TPE (Tree-structured Parzen Estimator) optimization ala Optuna
\end{itemize}

\subsection{Model validation}
\begin{enumerate}
\item \{lauerReich20\}: Infectious Disease Forecasting for Public Health
\cite{lauerReich20}
\item \{bergmeir18\}:A note on the validity of cross-validation for evaluating
autoregressive time series prediction \cite{bergmeir18}
\item \{bergmeirBenitez12\}: On the use of cross-validation for time series
predictor evaluation \cite{bergmeirBenitez12}
\end{enumerate}

\subsection{Integration of genomic data: molecular epidemiology}
\begin{itemize}
\item Rockett \cite{Rockett:2020uf}
\begin{enumerate}
\item examine the added value of near real-time genome sequencing of SARS-CoV-2
in a subpopulation of infected patients during the first 10 weeks
of COVID-19 containment in Australia and compare findings from genomic
surveillance with predictions of a computational agent-based model
(ABM)
\item based on AceMOD
\item 21 January and 28 March 2020, 1,617 cases of COVID- 19 were diagnosed
and reported to the NSW Ministry of Health. All patients resided in
metropolitan Sydney. 
\end{enumerate}
\end{itemize}

\section{Conclusions}

\subsection{Lessons learned}
\begin{itemize}
\item Be careful with standardization of country naming and data:
\begin{itemize}
\item Note excluded countries from the ECDC lists: TWN and HNG
\item cf \texttt{covasim. normalizeCVAgeData()}
\end{itemize}
\item Select which parameters you choose to calibrate, and their initial
bounds, carefully.
\item Differences in Covasim value/fit < 100 are not significant; larger
differences may not be either.
\end{itemize}

\subsection{Honest priors}

Efforts towards placing models like the ones considered here on firm
statistical foundations require better answers to a number of questions:
\begin{itemize}
\item How do the bounds established on global optimization, and artifacts
of the global search process, impact our understanding of the priors
they impose on nodel results?
\item How best to capture the sensitivity of our results given uncertainty
in testing rates?
\item What accuracy can we claim for model parameter estimation? 
\end{itemize}
Promising approaches include Approximate Bayesian Computation (ABC)
and history matching \cite{mckinley2018}.

\subsection{Scientific sharing, publishing and open source models}

\begin{quotation}

Early models relied on sparse, sometimes unreliable, data, and modelers
did not anticipate the emergence of important new facts.... For use
in an emergency, models developed through basic research need to be
\textquotedblleft operationalized\textquotedblright ---that is, made
robust for evaluating specific policy interventions. \textquotedblleft Nowcasting\textquotedblright{}
requires models that integrate incomplete, real-time data and emerging
medical knowledge to provide situational awareness.... Models must
also incorporate behavioral responses to policy interventions that
may change the course of an epidemic. \cite{Press1015}

\end{quotation}

Traditional compartmental models represent the core of epidemiological
prediction efforts, but become constrained as more fine-grained compartments
are considered. Models like Covasim bring the expressive power of
agent-based desciptions of behaviors over networks to allow evaluation
of many varieties of intervention strategies and empirical testing
against many data sources, but are more difficult to make statistical
inferences about. 

The well-engineered, open source Covasim codebase allows independent
model components to be investigated and incorporated separately. Basic
agent-based models can be related directly to compartment model analogs.
Careful experimentation building from these may be able to extend
the results into mathematically intractable regimes. The publishing
of full model implementation details, like that included with the
Panovska-Griffiths publication\cite{jpg20} and \href{https://github.com/Jasminapg/Covid-19-Analysis}{code repository}
is an excellent example. The COVID-19 pandemic, and others to come,
demand fast-paced scientific sharing that is catalyzed by such interactions.

\pagebreak{}

\bibliographystyle{plain}
\bibliography{covid}

\end{document}

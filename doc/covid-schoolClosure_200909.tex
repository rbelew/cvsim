%% LyX 2.3.0 created this file.  For more info, see http://www.lyx.org/.
%% Do not edit unless you really know what you are doing.
\documentclass[english]{achemso}
\usepackage[T1]{fontenc}
\usepackage[latin9]{inputenc}
\usepackage{babel}
\usepackage{amsmath}
\usepackage{amsthm}
\usepackage{graphicx}
\usepackage{setspace}
\doublespacing
\usepackage[unicode=true,pdfusetitle,
 bookmarks=true,bookmarksnumbered=false,bookmarksopen=false,
 breaklinks=false,pdfborder={0 0 1},backref=false,colorlinks=false]
 {hyperref}
\hypersetup{
 urlcolor=black}

\makeatletter

%%%%%%%%%%%%%%%%%%%%%%%%%%%%%% LyX specific LaTeX commands.

\title{Capturing 35 national school closure interventions in a model of
COVID-19 diagnoses and deaths }

%%%%%%%%%%%%%%%%%%%%%%%%%%%%%% Textclass specific LaTeX commands.
\numberwithin{equation}{section}

%%%%%%%%%%%%%%%%%%%%%%%%%%%%%% User specified LaTeX commands.
\usepackage{longtable}

\author{Richard K. Belew}
\email{rbelew@ucsd.edu}
\affiliation{Univ. California -- San Diego\\
La Jolla CA 92093}

\author{Cliff Kerr}
\email{ckerr@idmod.org}
\affiliation{Institute for Disease Modeling, Bellevue, WA}

\author{Jasmina Panovska-Griffiths}
\email{j.panovska-griffiths@ucl.ac.uk}
\affiliation{University College London}

\author{Dina Mistry}
\email{dmistry@idmod.org}
\affiliation{Institute for Disease Modeling, Bellevue, WA}

\SectionNumbersOn

\pagestyle{myheadings}
\markboth{Modelling school closures - Draft 9 Sept 20}{Modelling school closures - Draft 9 Sept 20}

\DeclareBibliographyCategory{cited}
\AtEveryCitekey{\addtocategory{cited}{\thefield{entrykey}}}

\usepackage{filecontents}

\nocite{*}

\makeatother

\begin{document}
\noindent \begin{center}
\emph{Draft: 9 Sept 20}
\par\end{center}
\begin{abstract}
An individual-based model called Covasim has recently been fit to
data regarding confirmed cases and deaths experience in the United
Kingdom during the first half of 2020, and then used it to evaluate
alternative intervention strategies there \cite{jpg20}. We extend
this methodology to consider data from 35 other countries, and use
a database of international intervention specifics called the \href{https://github.com/amel-github/covid19-interventionmeasures}{Covid-19 Control Strategies List}
to retrospectively model interventions employed in these countries.
Because the age distribution of populations is a key feature of the
COVID-19 pandemic and contacts among young people may play an especially
important role, we focus here on school closure interventions. 

Individual countries varied considerably in both the dates on which
they imposed school closings, and in the levels (kindergarten, primary,
secondary, university) specified. Critically, the age-stratefied sub-populations
supported by Covasim allow fine-grained specification of just which
individuals are affected by school closures at each educational level.
Simulations were first run for each of the 35 countries, without the
intervention, and data on confirmed cases and deaths was used to fit
key Covasim parameters. Next, specific intervention strategies employed
by each country were converted into specifications for Covasim, and
the same simulation parameters fit to a second model with the interventions
considered. In the 10 countries where there was a significant difference
between models, those incorporating school closures were considerably
better fits than those without. Since both models' parameters are
optimized and evaluated using the same criterion, improved fit with
the intervention model may be taken as evidence that the modeled intervention
is useful, at least in these 10 countries, in describing observed
data.
\end{abstract}

\section{Introduction}

\paragraph*{\{pearce20\}: Comparisons between countries are essential for the
control of COVID-19}

\begin{quotation}

Although international comparisons are often disparaged because of
different data quality and fears of the \textquoteleft ecological
fallacy\textquoteright , if done carefully they can play a major role
in our learning what works best for controlling COVID-19. \cite{10.1093/ije/dyaa108}

... the COVID-19 epidemic shows the need for epidemiology to go back
to its roots---thinking about populations. Studying disease occurrence
by person, place and time (often referred to as \textquoteleft descriptive
epidemiology\textquoteright ) is usually taught in introductory courses,
even if this approach is then paid little attention subsequently.
COVID-19 is a striking example of how we can learn a great deal from
comparing countries, states, regions, time trends and persons, despite
of all the difficulties. \cite{10.1093/ije/dyaa108}

\end{quotation}

With travel restrictions as they are in 2020, levels of migration
across national borders are considerably smaller than that across
state or provincial boundries. In some countries, school closures
were ordered only within particular states. For these reasons, and
given the data currently available, only models at the level of individual
countries and interventions ordered nationally are considered here. 

\paragraph*{\{Jewell20\}: Predictive Mathematical Models of the COVID-19 Pandemic:
Underlying Principles and Value of Projections}

\begin{quotation}

Predictive models for large countries, such as the US, are even more
problematic because they aggregate heterogeneous subepidemics in local
areas.... Models should also seek to use the best possible data for
local predictions. \cite{10.1001/jama.2020.6585}

\end{quotation}

\section{Results}

For each country the goal of Covasim optimization is to minimize the
difference between the model and data about the number of diagnosed
cases of, and deaths caused by, COVID. The objective measure used
for optimization is to minimize sum-squared-difference (SSD) ``fit
value'' between both of these. As data regarding deaths is believed
to generally more accurate than that about positive tests, SSD over
death rates is weighted twice that of diagnoses. 

Figure \ref{fig:All-country-statistics} shows the results for all
35 countries. Visual inspection of the model vs. data curves (see
below) suggests that fit values > 1000 reflect poor model fit; these
have been highlighted for both plain and school closure conditions.
Here we are interested in the difference between two models, one incorporating
school closure interventions and one without these. Figure \ref{fig:Improved-fit}
shows the range of differences in the fit value between these two
conditions. These differences can be very large when both models have
poor fit (Taiwan, North Macedonia, Hungary). For most countries, the
difference is small, < 150. In the 10 countries where there was a
significant difference between models, those incorporating school
closures were considerably better fits than those without.\footnote{This is the sentence that needs most statistical work around ``significant''
and ``better fit''!}

\begin{figure}
\noindent \begin{centering}
\includegraphics[width=0.8\paperwidth]{figs/comp-no:intrvn_200902}
\par\end{centering}
\caption{\label{fig:All-country-statistics}All country statistics}
\end{figure}

\begin{figure}
\noindent \begin{centering}
\includegraphics[width=0.75\paperwidth]{figs/improvedFit-annote}
\par\end{centering}
\caption{\label{fig:Improved-fit}Improved fit}
\end{figure}

Figure \ref{fig:Sample-countries} shows the curves for Malaysia and
New Zealand, two countries with large improvements in model fit, and
Switzerland which demonstates a smaller difference in fit. The Malaysian
model with school closures uses a smaller value for beta (3.000E-3
vs. 3.004E-3) and a much larger initial number of infected (4713 vs.
112); New Zealand shows a similar adjustment.

\begin{figure}
\noindent \begin{centering}
\includegraphics[height=0.8\paperheight]{figs/sampleCountries}
\par\end{centering}
\caption{\label{fig:Sample-countries}Sample countries}
\end{figure}


\section{Methods}

\subsection{Data sources}

\begin{figure}
\noindent \begin{centering}
\includegraphics[height=3in]{figs/dataSources}
\par\end{centering}
\caption{\label{fig:Data-sources}Data sources}
\end{figure}

\begin{itemize}
\item OurWorldInData \cite{owidcoronavirus}
\item \href{https://github.com/amel-github/covid19-interventionmeasures}{Covid-19 Control Strategies List},
developed by Am�lie Desvars-Larrive and colleagues {[}\href{https://www.csh.ac.at}{Complexity Science Hub Vienna}{]} 
\end{itemize}
\begin{figure}
\noindent \begin{centering}
\includegraphics[height=3in]{figs/educ_intervene-uniq}
\par\end{centering}
\caption{\label{fig:Intervention-dates-and}Intervention dates and educational
levels}

\end{figure}


\subsection{Covasim}

Covasim \cite{Kerr2020.05.10.20097469}

Optuna \cite{10.1145/3292500.3330701}

Tree-Structured Parzen Estimator \cite{10.1145/3377930.3389817}

CMAEvolution \cite{hansen2016cma}

Did not make use of SynthPop

\subsection{Key details}
\begin{itemize}
\item \texttt{self.weights = sc.mergedicts(\{'cum\_deaths':10, 'cum\_diagnoses':5\},
weights)}
\item \texttt{educLevelSize = \{'h':4, 'w':20, 'c':20, 'sk': 20, 'sp': 20,
'ss': 40 , 'su': 80\}}
\item Only countries with populations > 1e6 were considered.
\item Only countries with educational interventions coming after at least
one week of diagnoses and death data were considered. For example,
while both Ghana and Mauritius had school closures, but data on diagnoses
and deaths was not available for a week before the dates of their
closures.
\item Simulation start date picked when number of infections becomes > 50
\item Each educational level (kindergarten, primary, secondary, university)
was treated as a separate ``layer'' by Covasim and captured into
separate age ranges within the total population. This allows interventions
at different educational levels to be treated independently. 
\item Each school closure at any educational level is modeled as a simple
on/off: school closings start on a specified date, infection rate
was set very low (\texttt{ClosedSchoolBeta = 0.02}) for children in
each age group associated with the educational level, and are assumed
to be in effect until an end-of-closure date.
\item In a small number (6 of 35) of cases, the Covid-19 Control Strategies
List contained a specific \emph{end} for the school closure; for these
countries that date was used to end the intervention. Otherwise, a
default \texttt{SchoolOutDate = 2020-06-01} was used.
\end{itemize}

\subsection{Age distribution}
\begin{itemize}
\item \texttt{educLevel\_ages = \{'sk': {[}4,5{]}, 'sp': {[}6,10{]}, 'ss': {[}11,18{]},
'su': {[}19,24{]}\}}
\end{itemize}
The distinguished age cohorts associated with school levels is shown
in Figure 

\begin{figure}
\noindent \begin{centering}
\includegraphics[width=0.8\paperwidth]{figs/Malaysia-people}
\par\end{centering}
\caption{\label{fig:PopAgeDist}Population age distribution}
\end{figure}


\subsection{Estimating background testing rates}

All models of an epidemic that attempt to account for positive diagnoses
must have data, or make assumptions, about the context of \emph{testing}
within which the positive results are obtained. For example, Covasim
has different parameters specifying the assumed testing levels of
symptomatic vs. non-symptomatic individuals. Following a suggestion
from Roser et al:

\begin{quotation}

Perhaps the biggest challenge in thinking about testing is that the
number of tests performed depends on a country\textquoteright s testing
strategy---that is, how many people a country intends to test given
its context.... Some testing strategies focus on high-risk groups
such as health care workers or high-risk locations such as nursing
homes. Strategies that focus on those at the highest risk will result
in a lower number of tests performed per confirmed case, meaning countries
will know less about of the true magnitude of the outbreak in the
community at large.... In general, a high number of \emph{tests per
death} is preferable because it indicates widespread testing and assessment
of community transmission. \cite{owidcoronavirus}, identify-covid-exemplars

\end{quotation}

Our model bases each country's assumed testing rate on data regarding
tests per death. Most countries captured in OWID data report the number
of \emph{tests performed}, but a smaller fraction (8 of 35) report
instead the number of \emph{individuals tested}\footnote{https://ourworldindata.org/coronavirus\#acknowledgements}.
Relating one or the other to reported deaths and to a country's population
gives a rate of testing across time. Figure \ref{fig:Testing-rates}
shows the \emph{final}, most recent rate of testing for each country. 

\begin{figure}
\noindent \begin{centering}
\includegraphics[width=0.8\paperwidth]{figs/testRates}
\par\end{centering}
\caption{\label{fig:Testing-rates}Testing rates}
\end{figure}

We then make the optimistic assumption that this final level of testing
has been going on constantly throughout 2020. That is, we take the
final per-country tests-per-death measure as an \emph{upper bound}
on how much testing has actually occurred, and from which the diagnosis
data is drawn. This is certainly a crude assumption, and optimistically
imagines an abundance of testing providing statistical stability for
observed positive diagnoses. Nevertheless, it does allow per-country
levels of testing to be varied reasonably. As Figure \ref{fig:Testing-rates}
shows, these per-country estimates vary over two orders of magnitude,
but the estimates based on individual vs. on test reported data are
similar.

\subsection{Model optimization using Optuna}

We optimize over two parameters, $\beta$ and $\mathrm{\mathbf{initialInfect}}$,
the date of initial infection. Both TFE and CMAES samplers were used,
and NTrial=100 trials were allocated to find the search for parameter
values causing the model to best fit the data.

\section{Next steps}

\subsection{Model validation}
\begin{enumerate}
\item \{lauerReich20\}: Infectious Disease Forecasting for Public Health
\cite{lauerReich20}
\item \{bergmeir18\}:A note on the validity of cross-validation for evaluating
autoregressive time series prediction \cite{bergmeir18}
\item \{bergmeirBenitez12\}: On the use of cross-validation for time series
predictor evaluation \cite{bergmeirBenitez12}
\end{enumerate}

\subsection{Integration of genomic data: molecular epidemiology}
\begin{itemize}
\item Rockett \cite{Rockett:2020uf}
\begin{enumerate}
\item examine the added value of near real-time genome sequencing of SARS-CoV-2
in a subpopulation of infected patients during the first 10 weeks
of COVID-19 containment in Australia and compare findings from genomic
surveillance with predictions of a computational agent-based model
(ABM)
\item based on AceMOD
\item 21 January and 28 March 2020, 1,617 cases of COVID- 19 were diagnosed
and reported to the NSW Ministry of Health. All patients resided in
metropolitan Sydney. 
\end{enumerate}
\end{itemize}
\pagebreak{}

\bibliographystyle{plain}
\bibliography{covid}

\end{document}
